\documentclass{article}
\usepackage[utf8]{inputenc}
\usepackage{polski}
\usepackage{geometry}
\usepackage{pdfpages}
\usepackage{pdfpages}
\usepackage{listings}
\usepackage{listingsutf8}
\usepackage{multirow}
\usepackage{siunitx}
\usepackage{multirow}
\usepackage{booktabs}
\usepackage{tabularx}
\usepackage{placeins}
\usepackage{pdflscape}
\usepackage{graphicx}
\usepackage{subfig}
\usepackage{hyperref}
\usepackage{amsmath}
\usepackage{colortbl}

\geometry{
a4paper,
total={170mm,257mm},
left=20mm,
top=20mm
}
\newcolumntype{Y}{>{\centering\arraybackslash}X}
\renewcommand\thesection{}
\lstset{%
literate=%
 {ą}{{\k{a}}}1
 {ę}{{\k{e}}}1
 {Ą}{{\k{A}}}1
 {Ę}{{\k{E}}}1
 {ś}{{\'{s}}}1
 {Ś}{{\'{S}}}1
 {ź}{{\'{z}}}1
 {Ź}{{\'{Z}}}1
 {ń}{{\'{n}}}1
 {Ń}{{\'{N}}}1
 {ć}{{\'{c}}}1
 {Ć}{{\'{C}}}1
 {ó}{{\'{o}}}1
 {Ó}{{\'{O}}}1
 {ż}{{\.{z}}}1
 {Ż}{{\.{Z}}}1
 {ł}{{\l{}}}1
 {Ł}{{\l{}}}1
}

\title{Teoria współbieżności\\ 
Laboratorium V}
\author{Maciej Trątnowiecki}
\date{AGH, Semestr Zimowy, 2020}

\begin{document}
    \maketitle
    \section{Implementacja}
        Implementacja rozwiązania zakłada podział zadania generowania zbioru Mandelbrota na prostokątne fragmenty o szerokości pełnego obrazu i równej wysokości. Ilość fragmentów definiowana jest przez zmienną sectionCount, na jej podstawie dobierana jest wysokość każdego fragmentu. Program mierzy czas generacji obrazu w milisekundach.\\ 
        Program nie implementuje synchronizacji funkcji setRGB, jako że nie jest to potrzebne.
        
    \section{Pomiary czasu wykonania programu}
        Program przetestowano dla poniższej konfiguracji generatora.
        \begin{itemize}
            \item Szerokość obrazu: 850
            \item Wysokość obrazu: 800
            \item Liczba iteracji: 1000
            \item Przybliżenie: 300
        \end{itemize}
        Dla takich ustawień średni czas generacji na pojedynczym wątku mojego procesora wynosi 790 milisekund.\\
        Mój procesor wyposażony jest w 4 rdzenie i 8 wątków. \\
        Poniżej zamieszczam wyniki pomiarów. 
        
        \begin{center}
            \begin{table}[ht]
                \centering
                \begin{tabular}{|c|c|c|}
                    \hline
                    Liczba wątków w egzekutorze  & Liczba podproblemów & Średni czas wykonania (milisekundy) \\
                    \specialrule{1pt}{1pt}{1pt}
                    1 & 1 & 790 \\
                    \hline
                    4 & 4 & 450 \\
                    \hline
                    8  & 4 & 450 \\
                    \hline 
                    8 & 8 & 360 \\
                    \hline
                    8 & 16 & 240\\
                    \hline
                    8 & 80 & 210\\
                    \hline
                    8 & 800 & 220\\
                    \hline 
                    16 & 16 & 240\\
                    \hline
                    16 & 32 & 220\\
                    \hline
                    16 & 160 & 210\\
                    \hline
                    \end{tabular}
                \caption{Pomiar czasów wykonania}
                \label{tab:my_label}
            \end{table}
        \end{center}\\
        Na podstawie przeprowadzonych pomiarów stwierdzam że optymalne wyniki otrzymałem dla liczbie wątków egzekutora równej ilości wątków procesora, oraz liczby podproblemów równej jej dziesięciokrotności.  
\end{document}